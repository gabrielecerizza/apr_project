\documentclass[conference]{IEEEtran}
\IEEEoverridecommandlockouts
% The preceding line is only needed to identify funding in the first footnote. If that is unneeded, please comment it out.
\usepackage{cite}
\usepackage{amsmath,amssymb,amsfonts}
\usepackage{algorithmic}
\usepackage{graphicx}
\usepackage{textcomp}
\usepackage{xcolor}
\usepackage{multirow}
\def\BibTeX{{\rm B\kern-.05em{\sc i\kern-.025em b}\kern-.08em
    T\kern-.1667em\lower.7ex\hbox{E}\kern-.125emX}}
\begin{document}

\title{Speaker Recognition in the Wild\\}

\author{\IEEEauthorblockN{Gabriele Cerizza}
\IEEEauthorblockA{\textit{Department of Computer Science} \\
\textit{University of Milan}\\
Milan, Italy \\
gabriele.cerizza@studenti.unimi.it}
}

\maketitle

\begin{abstract}
Speaker recognition is the task of recognizing a person from their utterances, either to classify or to verify the identity of a given speaker. In this work, we compare the performance of different neural network models on this task, using a dataset collected under real-world conditions.  
\end{abstract}

\begin{IEEEkeywords}
speaker recognition, speaker identification, speaker verification, convolutional neural networks, attention
\end{IEEEkeywords}

\section{Introduction}

A challenging task in the field of audio pattern recognition is the one referred to as ``speaker recognition''. This task includes two problems: speaker identification and speaker verification~\cite{kabir2021survey,hajibabaei2018unified,chung2020defence}. 

Speaker identification is a closed-set problem in which the model has to identify the person to whom a given utterance belongs. The model is trained on utterances from each of the speakers, rendering this a classification problem. Speaker verification aims at validating whether a given utterance belongs to a given person. In academic research, this is treated as an open-set problem in which the model has to determine if two utterances belong to the same person. During evaluation, the model is presented with pairs of utterances belonging to speakers not included in the training set.

Speaker recognition can be further categorized as text-dependent or text-independent, in relation to the fact that the content of the utterances is fixed or not.

Speaker recognition applications encompass security control, online banking, retrieval, forensic tools and human computer interaction systems. A speaker recognition model can be also employed as part of a speaker diarization pipeline \cite{chung2018voxceleb2,park2022review}.

The focus of our research is text-independent speaker recognition. We compare the performance of different deep neural network models, which allow us to tackle both speaker identification and verification at the same time. To train and evaluate the models, we used data from VoxCeleb1~\cite{nagrani2020voxceleb}. These data were collected ``in the wild'', that is under real-world conditions where speech segments may be affected by the environment, like background noise and cross-talk.

The remainder of the report is organized as follows:


\section{Related works}

\subsection{Model Architectures}

Speaker recognition methods have been widely studied in literature. In particular, the interest in the field has grown in recent years, owing to the availability of large-scale datasets and to the increased computational power afforded by GPUs, which allows us to train deep neural networks.

Over the years, a particular approach became predominant. The idea is to produce fixed-sized feature representations of speech segments, which can then be processed by a classifier or directly compared. This approach was pioneered by i-vectors, obtained by joint factor analysis~\cite{dehak2011ivectors}. With the advent of neural networks, better speaker representations were obtained as the output of a bottleneck layer trained for classification. These speaker representations, known as embeddings, proved to be more robust with respect to acoustic conditions and more discriminating with respect to speakers, thus yielding more accurate results. Examples of such neural embeddings include d-vectors~\cite{variani2014dvectors} and, more recently, x-vectors~\cite{snyder2017deep,snyder2018xvectors}.

To perform speaker recognition with neural networks, an architecture composed of three blocks is traditionally employed~\cite{okabe2018asp}. The first block receives as input acoustic features such as MFCCs or Mel spectrograms and produces frame-level features. Typical layers of this block include CNN, Time-Delay Neural Network (TDNN)~\cite{peddinti2015timedelay,waibel1989timedelay}, mostly implemented as dilated convolution, and recurrent neural networks such as Long Short-Term Memory (LSTM). Popular architectures for the first block are borrowed from the image recognition domain, such as ResNet~\cite{he2016resnet}, VGG~\cite{simonyan2014vgg} and, recently, visual transformers like CoAtNet~\cite{dai2021coatnet}. 

The second block is required to aggregate the frame-level features, which have varying length depending on the length of the speech segment, into fixed dimensional utterance-level features. This block usually consists in a fully-connected layer stacked on top of a pooling layer. Many different pooling layers have been developed: from simply taking the average over the temporal dimension to sophisticated attention mechanisms~\cite{cai2018exploring}. The output of this block is the aforementioned embedding.

Finally, the third block contains the loss function, possibly preceded by a fully-connected layer projecting the embedding into a dimension whose size is equal to the number of speakers. Since the network is usually trained for multi-class classification, the loss function mixes and matches a plethora of Softmax variants with Cross-entropy loss.

It is worth to note that models, especially in competitions like the VoxCeleb Speaker Recognition Challenge\footnote{https://www.robots.ox.ac.uk/$\sim$vgg/data/voxceleb/competition2021.html}, are often combined using fusion or ensemble techniques.

\subsection{Speaker Verification Peculiarities}

\paragraph{Scoring}While neural networks trained for speaker verification are fundamentally identical to those trained for speaker identification, a further step is necessary to perform speaker verification. During evaluation, embeddings are extracted for each utterance. These embeddings are then compared pairwise. These comparisons may be carried out with backend systems like Probabilistic Linear Discriminant Analysis (PLDA), which can be trained on a different dataset with respect to the neural network~\cite{snyder2017deep}. However, more recently, neural embeddings have been compared simply by way of cosine similarity, achieving gains in performance in relation to PLDA~\cite{cai2018exploring}.

\paragraph{Score normalization}To reduce within trial variability and produce well calibrated and reliable scores, normalization is often applied. A popular score normalization technique is ``Adaptive S-Norm'' (AS-Norm)~\cite{matejka2017asnorm,cumani2011comparison}.

Let $s(\cdot,\cdot)$ be the score between two embeddings, $\mathcal{E}^{\text{top}}_e$ be the cohort of the $N$ closest embeddings to the enrollment utterance $e$ and $\mathcal{E}^{\text{top}}_t$ be the cohort of the $N$ closest embeddings to the test utterance $t$. Furthermore, let 
\begin{equation}
    S_e(\mathcal{E}^{\text{top}}_e) = \{(s,\varepsilon)\}_{\forall\varepsilon \in \mathcal{E}^{\text{top}}_e}~,
    S_e(\mathcal{E}^{\text{top}}_t) = \{(s,\varepsilon)\}_{\forall\varepsilon \in \mathcal{E}^{\text{top}}_t}~. 
\end{equation}
The normalized score is computed as
\begin{multline}
    s(e,t) = \frac{1}{2} \cdot \Bigg(\frac{s(e,t) - \mu(S_e(\mathcal{E}^{\text{top}}_e))}{\sigma(S_e(\mathcal{E}^{\text{top}}_e))} + \\
    \frac{s(e,t) - \mu(S_e(\mathcal{E}^{\text{top}}_t))}{\sigma(S_e(\mathcal{E}^{\text{top}}_t))}\Bigg)
    ~.
\end{multline}

\paragraph{Metrics}Once the similarity score between each pair of embeddings has been computed, we need to evaluate how good the model is in deciding whether the speech segments belong to the same speaker. To this end, various metrics have been proposed. Two notable metrics are Equal Error Rate (EER) and Minimum Detection Cost Function (MinDCF)~\cite{brummer2013bosaris,nist2018}. 

EER is defined as the point on the Receiver Operating Characteristic (ROC) curve in which $P_{\text{miss}} = P_{\text{fa}}$, where $P_{\text{miss}}$ is the ratio of samples belonging to the same speaker classified as dissimilar and $P_{\text{fa}}$ is the ratio of dissimilar samples classified as belonging to the same speaker. EER can be considered as a summary of the ROC curve.

On the other hand, MinDCF is derived from the normalization of the following weighted sum of misses and false-alarm error probabilities for a given decision threshold $\theta$:
\begin{equation}
    C_{\text{miss}} \times P_{\text{target}} \times P_{\text{miss}}(\theta) +
    C_{\text{fa}} \times (1 - P_{\text{target}}) \times P_{\text{fa}}(\theta) ,
\end{equation}
where $C_{\text{miss}}$ and $C_{\text{fa}}$ are respectively the cost of misses and false-alarms and $P_{\text{target}}$ is the \textit{a priori} probability of the specific speaker. The costs are usually fixed to 1. We refer to~\cite{nist2018} for the mathematical details of the normalization operation.

\paragraph{Pairwise Losses}Another difference between speaker identification and recognition lies in the choice of loss functions. Some models employ pairwise losses such as triplet loss or contrastive loss~\cite{cai2018exploring,chung2019delving}. The drawback of these losses is that they are notoriously difficult to train with and require careful fine-tuning~\cite{nagrani2020voxceleb}. Most state-of-the-art models now adopt Softmax variants, discussed below.

\subsection{Pooling Layers}

\paragraph{Temporal Average Pooling (TAP)}This is the simplest pooling, which takes the mean of the features along the time domain~\cite{cai2018exploring,chung2019delving}. In this way, all frames equally contribute to the utterance representation.

\paragraph{Self-Attentive Pooling (SAP)}Under the assumption that not all frames are equally informative, SAP uses an attention mechanism to learn which weight to assign to each frame. The frames are then multiplied by their respective weights and summed to obtain the utterance-level representation~\cite{cai2018exploring}.

More formally, let $W$, $b$ and $\mu$ be learnable parameters. Additionally, let $\{x_1,\dots,x_T\}$ be the time domain features of a given utterance. We first compute the attention weights $w_t$ as
\begin{equation}
    h_t = \text{tanh}(Wx_t + b)
\end{equation}
\begin{equation}
    w_t = \frac{\text{exp}(h_t^T\mu)}{\sum_{t=1}^T\text{exp}(h_t^T\mu)}~.
\end{equation}
Finally, we take the sum
\begin{equation}
    e = \sum_{t=1}^Tw_tx_t~.
\end{equation}

\paragraph{Other}A number of notable other pooling layers have been proposed. One of them is Self Multi-Head Attention Pooling, which splits the input to the layer into $N$ sequences, applies SAP to each of them and then concatenates the results~\cite{india2019selfmha}. Another one is Attentive Statistics Pooling (ASP), which combines SAP with mean and standard deviation statistics~\cite{okabe2018asp}.

\subsection{Softmax for Speaker Recognition}

Neural networks may be effectively trained with Softmax and Cross-entropy loss, which are traditionally employed for classification problems. 

The \textit{caveat} is that Softmax penalizes only classification errors, without enforcing intra-class compactness and inter-class separation~\cite{chung2020defence}. This makes Softmax unsuited for learning discriminative features, which map utterances belonging to the same speaker close to each other and far from utterances belonging to other speakers~\cite{liu2019large}. To remedy this, angular based losses have been developed, originally in the context of face recognition, and have now become widespread in the speaker recognition field, reportedly achieving an improved performance.

This class of loss functions is chiefly represented by Angular Additive Margin Softmax (AAM Softmax or ArcFace)~\cite{deng2019arcface}. This function introduces an angular margin penalty $m$, which forces the cosine similarity between the sample and its true class to be $m$ more than the cosine similarity between the sample and wrong classes. This difference is also multiplied by a scale factor $s$, which prevents the gradient from getting too small~\cite{chung2020defence,hajibabaei2018unified}.

Recently, Sub-center AAM Softmax (SC AAM Softmax) has been proposed~\cite{deng2020subarcface}. This loss function relaxes the intra-class compactness constraint by incorporating $K$ sub-centers for each class and forcing each sample to be close to any one of the positive sub-centers. It is expected that one dominant sub-center will contain the majority of ``clean'' data belonging to the class, while the hard and noisy samples will gravitate towards the non-dominant sub-centers. As such, SC AAM Softmax is more robust to noise.

\section{Data and Features}

\subsection{Dataset}

For our study, we elected to use the VoxCeleb1 dataset, which is characterized by speech segments collected ``in the wild'' from YouTube~\cite{nagrani2020voxceleb,chung2018voxceleb2,chung2019delving}. This dataset is particularly challenging due to the fact that the samples present both extrinsic and intrinsic variations. Extrinsic variations include background noise, chatter, music, laughter, cross-talk and varying room acoustics (from small indoor rooms to stadiums). Intrinsic variations concern the heterogeneity of age, gender and nationality of the speakers. 

The VoxCeleb1 dataset consists of over 150,000 utterances from 1,251 celebrities. However, due to hardware constraints, we were unable to take advantage of the full dataset and had to confine our experiments to a randomly sampled subset. We extracted all the utterances of 100 randomly chosen speakers, while retaining the gender proportions to guarantee the representativeness of the sample. We kept the official splits for the training, validation and test sets provided for identification. To evaluate the verification capabilities of the models, we used utterances from 10 further speakers not present in the subset previously described. Some statistics on both the dataset and the subsets are provided in Table~\ref{tab:dataset}.

\begin{table}[htbp]
    \caption{VoxCeleb1 Dataset Statistics}
    \begin{center}
        \begin{tabular}{|c|c|c|c|c|}
        \cline{3-5}
        \multicolumn{2}{c|}{\textbf{}} & \textbf{Full Dataset} & \textbf{Identif. Set} & \textbf{Verif. Set}\\
        \hline
        \multicolumn{2}{|c|}{\textbf{Speakers No.}} & 1,251 & 100 & 10\\
        \hline
        \multicolumn{2}{|c|}{\textbf{Samples No.}} & 152,123 & 12,123 & 1,896 \\
        \hline
        \multirow{2}{*}{\textbf{Gender}} & \textit{Male} & 0.55 & 0.51 & 0.50 \\
        & \textit{Female} & 0.55 & 0.52 & 0.50 \\
        \hline
        \multirow{4}{*}{\textbf{Nationality}$^{\mathrm{a}}$} & \textit{USA} & 0.65 & 0.52 & 0.41 \\
        & \textit{UK} & 0.21 & 0.12 & 0.32 \\
        & \textit{Italy} & 0.01 & 0.05 & 0.07 \\
        & \textit{Russia} & 0.07 & 0.02 & 0.02 \\
        \hline
        \multirow{2}{*}{\textbf{Seconds}} & \textit{Mean} & 8.12 & 7.59 & 5.67 \\
        & \textit{Std} & 2.55 & 3.55 & 4.55 \\
        \hline
        \multicolumn{4}{l}{$^{\mathrm{a}}$Only the four most frequent nationalities are listed.}
        \end{tabular}
        \label{tab:dataset}
    \end{center}
\end{table}

\subsection{Features}

As input features to our models, we extracted 80-dimensional log Mel spectrograms with a window length of 25 ms (Hamming window) and a frame-shift of 10 ms, to which we applied cepstral mean normalization at the instance level. As noted in the literature, MFCCs are ill suited for the VoxCeleb1 dataset, since they degrade with real-world noise and they lack speaker discriminating features like pitch information~\cite{nagrani2020voxceleb}.

Considering the ``in the wild'' nature of the dataset, in order to improve generalization, we also applied data augmentation. We performed four types of offline data augmentation. For each training sample we
\begin{enumerate}
    \item perturbed the waveform by a speed factor of 0.9 or 1.1, randomly chosen;
    \item added background noise, randomly chosen from the MUSAN dataset~\cite{snyder2015musan}, with a signal-to-noise ratio DB between 0 and 15;
    \item added babble effect by superimposing a speech track randomly chosen from MUSAN over the waveform, with a signal-to-noise ratio DB between 15 and 20;
    \item added reverberation by using the \texttt{pedalboard} library from Spotify\footnote{https://github.com/spotify/pedalboard}.
\end{enumerate}
As a consequence, we obtained a total of 59,140 training samples. Furthermore, we chained SpecAugment~\cite{park2019specaug} online data augmentation by randomly masking 0 to 5 frames in the time domain and 0 to 10 frequency bands. Validation and test samples were left untouched.

\subsection{K-Means Clustering}

For a better understanding of the complexity of the task, we show in Fig.~\ref{fig:k_means} the distribution of clusters identified by K-Means on the test set, after learning the location of the centers on the training set. To obtain a bidimensional representation of the data, we used Principal Component Analysis (PCA), again trained solely on the training set. Compared with the actual distribution of ground truth labels, it is clear how challenging the task is.

\begin{figure*}[htbp]
    \centerline{\includegraphics[width=0.7\textwidth]{img/k_means.png}}
    \caption{Comparison between K-Means clusters and ground truth labels on the speaker identification test set.}
    \label{fig:k_means}
\end{figure*}

\section{Models and Training}

\subsection{Models}

We compared three different neural network models, which we describe below.

\paragraph{ResNet34-SE}

The first model is based on the ResNet34 architecture~\cite{he2016resnet}, modified to accept Mel spectrograms as input. Following the approach described in~\cite{heo2020clova,chung2020defence}, we halved the number of channels in the convolutional layers and adopted SAP to aggregate features in the time domain. Additionally, we introduced Squeeze-and-Excitation (SE) layers~\cite{jie2020squeeze}, which proved to be good for speaker verification according to~\cite{zhang2021beijing}. For the loss function, we used SC AAM Softmax.

\paragraph{LAS-MHA}

Taking inspiration from the Listen, Attend and Spell (LAS)~\cite{irie2019las} model for sequence-to-sequence speech recognition and from the Multi-Head Attention (MHA)~\cite{vaswani2017attention} employed in language modeling, we devised a new architecture. We stacked a MHA layer with 8 heads on top of the encoder part of the LAS model, composed of CNN layers followed by bidirectional LSTM layers. We used SC AAM Softmax during training.

\paragraph{EfficientNetV2}

Many state-of-the-art architectures in speaker recognition have been borrowed from the computer vision domain. For this reason, we decided to use EfficientNetV2~\cite{tan2021efficientnetv2}, a recently developed model that performs well in image classification while being optimized for training speed and parameter efficiency. EfficientNetV2 is characterized by its use of depthwise convolution in place of traditional convolution, which allows to reduce the number of parameters. EfficientNetV2 was trained with standard Softmax loss.

\subsection{Training and Scoring}

Training was carried out in two steps, according to the Large Margin Fine-Tuning strategy~\cite{thienpondt2021voxsrc}. We first trained the models on random crops of the utterances with a length of 3 seconds (eventually padded), to avoid overfitting. When the models stopped learning, we fine-tuned them with random crops of 6 seconds. We also increased the margin $m$ of SC AAM Softmax from 0.1 to 0.3 and the scale $s$ from 15 to 30. Increasing the margin makes training harder, but leads to better embeddings. Additionally, we decreased the number of sub-centers from 3 to 1. The models were evaluated against 6 seconds random crops.

We used 256 dimensional embeddings, which were found to be the best for identification in~\cite{hajibabaei2018unified}.

For speaker verification, we compared the embeddings by cosine similarity and then normalized the scores with AS-Norm, using cohorts of size 100. The parameters of EER were defined as stated in~\cite{nist2018}.

\section{Results}

Considering the fact that we used a subset of VoxCeleb1 for both training and test set, a direct comparison with models described in literature is not possible. Besides, researchers often trained their models on the much larger VoxCeleb2 dataset before evaluating them on the VoxCeleb1 test set. For a better understanding of the performance of the models, we therefore added the kind of training set to the results shown in Table. For literature models we reported the verification results of the original (O) verification split.

\begin{table*}[htbp]
    \caption{VoxCeleb1 Test Set Results}
    \begin{center}
        \begin{tabular}{|c|c|c|c|c|c|c|c|}
        \hline
        \textbf{Model} & \textbf{Year} & \textbf{Training Set} &  \textbf{Top1 Acc.} & \textbf{Top5 Acc.} & \textbf{F1 Score} & \textbf{EER(\%)} & \textbf{MinDCF}$^{\mathrm{a}}$\\
        \hline
        Cai et al.~\cite{cai2018exploring} & 2018 & VoxCeleb1 & 89.9 & 95.7 & - & - & - \\
        Cai et al.~\cite{cai2018exploring} & 2018 & VoxCeleb1 & - & - & - & 5.27 & 0.439 \\
        Cai et al.~\cite{cai2018exploring} & 2018 & VoxCeleb1 & - & - & - & 4.46 & 0.577 \\
        Chung et al.~\cite{chung2018voxceleb2} & 2018 & VoxCeleb1 & - & - & - & 7.8 & 0.71 (0.01) \\
        Chung et al.~\cite{chung2018voxceleb2} & 2018 & VoxCeleb2 & - & - & - & 3.95 & 0.429 (0.01) \\
        Okabe et al.~\cite{okabe2018asp} & 2018 & VoxCeleb1 & - & - & - & 3.85 & 0.406 (0.01) \\
        Hajibabaei, Dai~\cite{hajibabaei2018unified} & 2018 & VoxCeleb1 & 94.6 & 98.1 & - & 4.69 & 0.453 (0.01) \\
        Hajibabaei, Dai~\cite{hajibabaei2018unified} & 2018 & VoxCeleb1 & 92.8 & 97.5 & - & 4.30 & 0.413 (0.01) \\
        Snyder et al.~\cite{snyder2018xvectors}$^{\mathrm{b}}$ & 2018 & VoxCeleb2 & - & - & - & 3.10 & 0.33 (0.01) \\
        Xie et al.~\cite{xie2019utterance} & 2019 & VoxCeleb2 & - & - & - & 3.22 & - \\
        Chung et al.~\cite{chung2019delving} & 2019 & VoxCeleb1 & 89.00 & 96.15 & - & 5.37 & - \\
        Chung et al.~\cite{chung2019delving} & 2019 & VoxCeleb1 & 89.00 & 95.94 & - & 5.26 & - \\
        Chung et al.~\cite{chung2020defence} & 2020 & VoxCeleb2 & - & - & - & 2.21 & - \\
        Desplanques et al.~\cite{desplanques2020ecapa} & 2020 & VoxCeleb2 & - & - & - & 0.87 & 0.1066 (0.01) \\
        Nagrani et al.~\cite{nagrani2020voxceleb} & 2020 & VoxCeleb2 & - & - & - & 2.87 & 0.31 (0.01) \\
        Xiang~\cite{xiang2020xx205} & 2020 & VoxCeleb2 & - & - & - & 0.7712 & 0.0774 (0.01) \\
        Xiang~\cite{xiang2020xx205} & 2020 & VoxCeleb2 & - & - & - & 0.8350 & 0.0638 (0.01) \\
        Hajavi, Etemad~\cite{hajavi2021siamese} & 2021 & VoxCeleb1 & - & - & - & 3.14 & - \\
        Thienpondt et al.~\cite{thienpondt2021voxsrc} & 2021 & VoxCeleb2 & - & - & - & 0.64 & 0.0700 (0.01) \\
        Thienpondt et al.~\cite{thienpondt2021voxsrc} & 2021 & VoxCeleb2 & - & - & - & 0.56 & 0.0743 (0.01) \\
        Zhao et al.~\cite{zhao2021speakin} & 2021 & VoxCeleb2 & - & - & - & 0.5249 & 0.0498 (0.01) \\
        Zhao et al.~\cite{zhao2021speakin} & 2021 & VoxCeleb2 & - & - & - & 0.5594 & 0.0480 (0.01) \\
        \hline
        \multicolumn{8}{l}{$^{\mathrm{a}}$If provided, we noted the $P_{\text{target}}$ value within parentheses.} \\
        \multicolumn{8}{l}{$^{\mathrm{b}}$Results reported in~\cite{nagrani2020voxceleb}.}
        \end{tabular}
        \label{tab:dataset}
    \end{center}
\end{table*}

\section{Conclusion}

Less data, more difficult to capture discriminative embeddings.


curriculum
Pipeline diagram
remarkable
Times new roman font in image
LAS encoder repurposed
VAD

first slide: speaker id, speaker ver: 1) imagine with utterance and several heads, who is the speaker? 2) two utterances: same speaker?

regarded
predominant 
devise

paradigm was to construct an utterance specific representation  



The IEEEtran class file is used to format your paper and style the text. All margins, 
column widths, line spaces, and text fonts are prescribed; please do not 
alter them. You may note peculiarities. For example, the head margin
measures proportionately more than is customary. This measurement 
and others are deliberate, using specifications that anticipate your paper 
as one part of the entire proceedings, and not as an independent document. 
Please do not revise any of the current designations.

\section{Prepare Your Paper Before Styling}
Before you begin to format your paper, first write and save the content as a 
separate text file. Complete all content and organizational editing before 
formatting. Please note sections  below for more information on 
proofreading, spelling and grammar.

Keep your text and graphic files separate until after the text has been 
formatted and styled. Do not number text heads---{\LaTeX} will do that 
for you.

\subsection{Abbreviations and Acronyms}\label{AA}
Define abbreviations and acronyms the first time they are used in the text, 
even after they have been defined in the abstract. Abbreviations such as 
IEEE, SI, MKS, CGS, ac, dc, and rms do not have to be defined. Do not use 
abbreviations in the title or heads unless they are unavoidable.

\subsection{Units}
\begin{itemize}
\item Use either SI (MKS) or CGS as primary units. (SI units are encouraged.) English units may be used as secondary units (in parentheses). An exception would be the use of English units as identifiers in trade, such as ``3.5-inch disk drive''.
\item Avoid combining SI and CGS units, such as current in amperes and magnetic field in oersteds. This often leads to confusion because equations do not balance dimensionally. If you must use mixed units, clearly state the units for each quantity that you use in an equation.
\item Do not mix complete spellings and abbreviations of units: ``Wb/m\textsuperscript{2}'' or ``webers per square meter'', not ``webers/m\textsuperscript{2}''. Spell out units when they appear in text: ``. . . a few henries'', not ``. . . a few H''.
\item Use a zero before decimal points: ``0.25'', not ``.25''. Use ``cm\textsuperscript{3}'', not ``cc''.)
\end{itemize}

\subsection{Equations}
Number equations consecutively. To make your 
equations more compact, you may use the solidus (~/~), the exp function, or 
appropriate exponents. Italicize Roman symbols for quantities and variables, 
but not Greek symbols. Use a long dash rather than a hyphen for a minus 
sign. Punctuate equations with commas or periods when they are part of a 
sentence, as in:
\begin{equation}
a+b=\gamma\label{eq}
\end{equation}

Be sure that the 
symbols in your equation have been defined before or immediately following 
the equation. Use 

\subsection{\LaTeX-Specific Advice}

Please use ``soft'' (e.g., \verb|\eqref{Eq}|) cross references instead
of ``hard'' references (e.g., \verb|(1)|). That will make it possible
to combine sections, add equations, or change the order of figures or
citations without having to go through the file line by line.

Please don't use the \verb|{eqnarray}| equation environment. Use
\verb|{align}| or \verb|{IEEEeqnarray}| instead. The \verb|{eqnarray}|
environment leaves unsightly spaces around relation symbols.

Please note that the \verb|{subequations}| environment in {\LaTeX}
will increment the main equation counter even when there are no
equation numbers displayed. If you forget that, you might write an
article in which the equation numbers skip from (17) to (20), causing
the copy editors to wonder if you've discovered a new method of
counting.

{\BibTeX} does not work by magic. It doesn't get the bibliographic
data from thin air but from .bib files. If you use {\BibTeX} to produce a
bibliography you must send the .bib files. 

{\LaTeX} can't read your mind. If you assign the same label to a
subsubsection and a table, you might find that Table I has been cross
referenced as Table IV-B3. 

{\LaTeX} does not have precognitive abilities. If you put a
\verb|\label| command before the command that updates the counter it's
supposed to be using, the label will pick up the last counter to be
cross referenced instead. In particular, a \verb|\label| command
should not go before the caption of a figure or a table.

Do not use \verb|\nonumber| inside the \verb|{array}| environment. It
will not stop equation numbers inside \verb|{array}| (there won't be
any anyway) and it might stop a wanted equation number in the
surrounding equation.

\subsection{Some Common Mistakes}\label{SCM}
\begin{itemize}
\item The word ``data'' is plural, not singular.
\item The subscript for the permeability of vacuum $\mu_{0}$, and other common scientific constants, is zero with subscript formatting, not a lowercase letter ``o''.
\item In American English, commas, semicolons, periods, question and exclamation marks are located within quotation marks only when a complete thought or name is cited, such as a title or full quotation. When quotation marks are used, instead of a bold or italic typeface, to highlight a word or phrase, punctuation should appear outside of the quotation marks. A parenthetical phrase or statement at the end of a sentence is punctuated outside of the closing parenthesis (like this). (A parenthetical sentence is punctuated within the parentheses.)
\item A graph within a graph is an ``inset'', not an ``insert''. The word alternatively is preferred to the word ``alternately'' (unless you really mean something that alternates).
\item Do not use the word ``essentially'' to mean ``approximately'' or ``effectively''.
\item In your paper title, if the words ``that uses'' can accurately replace the word ``using'', capitalize the ``u''; if not, keep using lower-cased.
\item Be aware of the different meanings of the homophones ``affect'' and ``effect'', ``complement'' and ``compliment'', ``discreet'' and ``discrete'', ``principal'' and ``principle''.
\item Do not confuse ``imply'' and ``infer''.
\item The prefix ``non'' is not a word; it should be joined to the word it modifies, usually without a hyphen.
\item There is no period after the ``et'' in the Latin abbreviation ``et al.''.
\item The abbreviation ``i.e.'' means ``that is'', and the abbreviation ``e.g.'' means ``for example''.
\end{itemize}
An excellent style manual for science writers is.

\subsection{Authors and Affiliations}
\textbf{The class file is designed for, but not limited to, six authors.} A 
minimum of one author is required for all conference articles. Author names 
should be listed starting from left to right and then moving down to the 
next line. This is the author sequence that will be used in future citations 
and by indexing services. Names should not be listed in columns nor group by 
affiliation. Please keep your affiliations as succinct as possible (for 
example, do not differentiate among departments of the same organization).

\subsection{Identify the Headings}
Headings, or heads, are organizational devices that guide the reader through 
your paper. There are two types: component heads and text heads.

Component heads identify the different components of your paper and are not 
topically subordinate to each other. Examples include Acknowledgments and 
References and, for these, the correct style to use is ``Heading 5''. Use 
``figure caption'' for your Figure captions, and ``table head'' for your 
table title. Run-in heads, such as ``Abstract'', will require you to apply a 
style (in this case, italic) in addition to the style provided by the drop 
down menu to differentiate the head from the text.

Text heads organize the topics on a relational, hierarchical basis. For 
example, the paper title is the primary text head because all subsequent 
material relates and elaborates on this one topic. If there are two or more 
sub-topics, the next level head (uppercase Roman numerals) should be used 
and, conversely, if there are not at least two sub-topics, then no subheads 
should be introduced.

\subsection{Figures and Tables}
\paragraph{Positioning Figures and Tables} Place figures and tables at the top and 
bottom of columns. Avoid placing them in the middle of columns. Large 
figures and tables may span across both columns. Figure captions should be 
below the figures; table heads should appear above the tables. Insert 
figures and tables after they are cited in the text. Use the abbreviation 

\begin{table}[htbp]
\caption{Table Type Styles}
\begin{center}
\begin{tabular}{|c|c|c|c|}
\hline
\textbf{Table}&\multicolumn{3}{|c|}{\textbf{Table Column Head}} \\
\cline{2-4} 
\textbf{Head} & \textbf{\textit{Table column subhead}}& \textbf{\textit{Subhead}}& \textbf{\textit{Subhead}} \\
\hline
copy& More table copy$^{\mathrm{a}}$& &  \\
\hline
\multicolumn{4}{l}{$^{\mathrm{a}}$Sample of a Table footnote.}
\end{tabular}
\label{tab1}
\end{center}
\end{table}

Figure Labels: Use 8 point Times New Roman for Figure labels. Use words 
rather than symbols or abbreviations when writing Figure axis labels to 
avoid confusing the reader. As an example, write the quantity 
``Magnetization'', or ``Magnetization, M'', not just ``M''. If including 
units in the label, present them within parentheses. Do not label axes only 
with units. In the example, write ``Magnetization (A/m)'' or ``Magnetization 
\{A[m(1)]\}'', not just ``A/m''. Do not label axes with a ratio of 
quantities and units. For example, write ``Temperature (K)'', not 
``Temperature/K''.

\section*{Acknowledgment}

The preferred spelling of the word ``acknowledgment'' in America is without 
an ``e'' after the ``g''. Avoid the stilted expression ``one of us (R. B. 
G.) thanks $\ldots$''. Instead, try ``R. B. G. thanks$\ldots$''. Put sponsor 
acknowledgments in the unnumbered footnote on the first page.

\section*{References}

Please number citations consecutively within brackets . Refer simply to the reference 
number, as in  except at 
the beginning of a sentence: ``Reference was the first $\ldots$''

Number footnotes separately in superscripts. Place the actual footnote at 
the bottom of the column in which it was cited. Do not put footnotes in the 
abstract or reference list. Use letters for table footnotes.

Unless there are six authors or more give all authors' names; do not use 
``et al.''. Papers that have not been published, even if they have been 
submitted for publication, should be cited as ``unpublished'' . 
Capitalize only the first word in a paper title, except for proper nouns and 
element symbols.

For papers published in translation journals, please give the English 
citation first, followed by the original foreign-language citation 
\cite{kabir2021}


\bibliographystyle{IEEEtran}
\bibliography{bibtex_entries}
\vspace{12pt}
\color{red}
IEEE conference templates contain guidance text for composing and formatting conference papers. Please ensure that all template text is removed from your conference paper prior to submission to the conference. Failure to remove the template text from your paper may result in your paper not being published.

\end{document}
